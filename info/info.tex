\initunifonts
\chyph                      % české vzory dělení slov
\fontfam[Pagella]
% \famvardef\tt{\Schola\setff{-liga;-tlig}\rm} %zmena url fontu

\margins/1 a4 (3,3,3,3)cm   % okraje 3cm ze všech stran pro stranu A4
\typosize[11/13.5]            % velikost písma 11pt na 13pt řádkování
\activettchar"              % znak " bude ohraničovat řádkový verbatim
\hyperlinks\Blue\Blue       % interní odkazy modré, webové odkazy zelené
% \outlines0                  % klikací obsah po straně PDF prohlížeče
\parindent=15pt
% \nopagenumbers
% \setmathsizes[11/8.4/6]\normalmath
\load[mte]\enablemte
\load[vlna]

\tit Scénář, semestrální práce BI-\TeX

Cílem semestrální práce by měl být připravený balíček maker pro formátování scénáře. 
Uživatel by měl být schopen soustředit se na samotné psaní textů, 
formátování by mělo probíhat jen za pomocí minimálního počtu předpřipravených příkazů.

\chap Jak vypadá scénář?

\sec Druhy scénářů

Ideální forma scénáře se značně odlišuje podle toho pro jaké médium je určena. Scénárista může k různým formám představení přistupovat odlišně.
Filmy se s magickou mocí střihu může odehrávat v desítkách různých prostředích a může mít připravené desítky různých prostředí. Divadlo oproti tomu musí být s 
proměnou scény značně komornější. Pro dramatičtější změnu scény je typicky třeba divadelní přestávky. \cite[scenarcz] 

\medskip\noindent Druhů scénáře tedy rozlišujeme několik:

\begitems
* Filmový
* Televizní
* {\bf Divadelní}
* Rozhlasový
* Komiksový
* Technický
* Další méně běžné druhy.
\enditems

\noindent  Filmy navíc scénářů můžou mít v průběhu tvorby několik. Rozdílný scénář totiž potřebuje herec a jiný zase kameraman. 
Zde se tedy zatím zaměříme na scénář {\bf divadelní}.

\sec Jak vypadá divadelní scénář?

\secc Úvodní stránky

Úvodní strana scénáře by měla obsahovat pouze čistě {\bf název hry} a {\bf jméno autora}. Případně do kraje stránky můžeme doplnit rok vzniku.

Dále může scénář obsahovat {\bf seznam postav} s krátkým popisem ať už vzhledu, charakteru, či role v ději. Obdobně pak může následovat {\bf seznam prostředí}, 
kde může být uvedený popis obrazů. Především tech, kde třeba dochází k dramatičtější úpravě scény.

Přiložený může být i jednoduchý obsah všech scén a dějství.

\secc Členění divadelní hry \cite[eichler]

Obvykle se divadelní hra člení do několika {\bf dějství} (případně {\it jednání, či aktů}). Zpravidla jich bývá do tří, maximálně pěti. Během střídání jsou obvykle přestávky
pro složitější přestavbu scény. Dějství jsou basicly dobrovolný. Typicky stačí scény.

Dějství se potom dělí do dalších menších navazujících celků, do {\bf scén či obrazů}. Mezi scénami
nedochází k přerušení.

Samotný děj potom posouvají především monology a dialogy postav. 
Ty společně s popisy scén a chování postav tvoří jádro obsahu scénáře.

\secc Způsob psaní

Samotné scény by potom měli být měli být přehledně formátovány. {\bf Tučně} můžeme zvýraznit nadpisy scén, či určení prostředí,
ve kterém se vyskytujeme. {\it Kurzívou} by pak mělo být psáno vše, co není přímá řeč. CAPSLOCKEM (případně tučně) píšeme jména postav.

\secc Dialogy

Dialogy (a monology) píšeme na stejný řádek jako jméno postavy. Za jménem může být dvojtečka. Řeč by také měla být od jména postavy patřičně odsazená, aby nehledě na délku
jména tvořila vlastní sloupec. Popisujeme-li chování právě řečnící postavy, pak popis můžeme vložit přímo do dialogu. Držíme se {\it kurzívy} a vložíme popis do závorek.

Popisujeme-li {\bf chování} jiné postavy, či cokoliv jiného, pak závorky vynecháme a vložíme událost na nový řádek. Zde nedodržujeme rozdělení na sloupce. Popis se roztáhne na celou stránku.
Typicky zarovnaný buď do prava, nebo centrované.

Pro přehlednost mezi replikami vynecháváme jeden řádek.  

\sec Standartni fonty 
Alespoň filmové scénáře se pevně drží fontu Courier12. \cite[indufont]

\chap Co tedy naprogramovat?

\sec Speciální stránky

\begitems
* Úvodní stránka - název, autor, datum
* Postavy - výpis postav s krátkým popisem
* Prostředí - výpis prostředí s krátkým popisem
* Obsah
\enditems

\sec Členění scénáře

\begitems
* Dějství
* Scény
\enditems

\sec Textová pole 

\begitems
* Popis scény osamoceně
* Popis scény v replice
* Replika - postava + text
\enditems

\chap Zdroje

\usebib/c (iso690) zdroje.bib

\bye